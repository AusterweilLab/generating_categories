%!TEX output_directory = latex_out/

\documentclass[12pt]{article}
\usepackage[letterpaper, margin=1in, headheight=15pt]{geometry}
\usepackage{setspace, amsmath}


\begin{document}

\section{The Exemplar Copy-And-Tweak Model}

This model assumes that people represent categories as a collection of stored observations. When prompted to generate new examples, they copy and randomly tweak one of the stored examples. Thus, the model's generation is a two-part process:

\begin{enumerate}
  \item Select a source example from memory.
  \item Generate an example that is similar-but-not-too-similar to the source.
\end{enumerate}

More formally, let $x$ be a $j$-exemplar by $k$-feature matrix, corresponding to the stored exemplars associated with the target category. The probability that a source example $z$ is selected is uniform across all exemplars. After a source exemplar $z$ is selected, similarity between candidate generation options $y$ is computed:

\begin{equation}
  s(y,z) = exp( -c \sum_k{|y_k - z_k|w_k})
\end{equation}

Where $c$ and $w_k$ are the standard specificity and attention weights discussed elsewhere. The goal of the model is to generate an item that is similar but sufficiently tweaked from the source. Thus, probability a candidate stimulus will be generated is given by:

\begin{equation}
    p(y|z)  = \dfrac
    { \exp \left\{\theta \cdot s(y,z) \right\} I\left(s(y,z) \leq \tau\right) }
    {\sum_i{\exp \left\{ \theta \cdot s(y_i,z) \right\} I\left(s(y_i,z) \leq \tau\right)}} 
    \label{eq:generation-choice}
\end{equation}
% 
where $\theta$ is a response determinism parameter, $\tau$ is the similarity threshold, and $I(\cdot)$ is the indicator function, which returns 1 when it is passed a true expression and 0 otherwise. So, $I\left(s(y,z) \leq \tau\right)$ is 1 when the candidate exemplar $y$ is tweaked enough. When $\tau=1$, the threshold has no effect.

To obtain predictions not depending on a given source example, the model's predictions can be aggregated over all possible sources:

\begin{equation}
  p(y) = \sum_z{p(z)p(y|z) }
\end{equation}




\end{document}
