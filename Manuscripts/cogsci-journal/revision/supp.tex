%!TEX output_directory = latex_out/

\documentclass[12pt]{article}
\usepackage[letterpaper, margin=1in, headheight=15pt]{geometry}
\usepackage{amsmath,fancyvrb}
\usepackage[natbibapa]{apacite}
\usepackage{pgfplots}
\usepackage{setspace}

% set up PGF
\pgfplotsset{compat=1.6}
\newcommand\inputpgf[2]{{
\let\pgfimageWithoutPath\pgfimage
\renewcommand{\pgfimage}[2][]{\pgfimageWithoutPath[##1]{#1/##2}}
\input{#1/#2}
}}

\begin{document}
\doublespacing
\section{Finding Contrast Effects in Prior Work}

Although existing accounts of category generation broadly overlook the role of
category contrast in determining what is novel versus familiar, it was
implicitly assumed that learners in previous experiments were {\em successful}
in creating new categories. Thus, if contrast plays a robust role in category
genration, its effects should be discernable within the experimental results of
these studies. To provide a test of the influence of category contrast within
existing data, we conducted a novel analysis of Experiment 3 from
\cite{jern2013probabilistic}.\footnote{Although \cite{jern2013probabilistic}
	reported four experiments, we focus on Experiment 3 because their first two
	experiments tested generation of items into known categories, and their fourth
	experiment was identical to Experiment 3 but with a restricted generation
	space.}

Participants in their experiment were exposed to members of two
experimenter-defined categories of `crystals' varying in hue, saturation, and
size. Each category possessed a unique hue, but varied in saturation and size:
In the `Positive' condition, there was a positive correlation between these
features (i.e., larger sized crystals were more saturated), and in the
`Negative' condition this relation was reversed. In the `Neutral' condition,
there was no correlation between saturation and size. After learning about the
categories from each condition, participants were asked to generate six
exemplars belonging to a novel class. As noted above,
\cite{jern2013probabilistic} found that the generated categories tended to
follow the distributional properties of the experimenter-defined categories:
Generated categories were tightly distributed along the hue feature, and
possessed the same saturation-size correlations as in the learned categories.
\cite{jern2013probabilistic}, however, did not analyze or discuss how the
generated categories {\em differed} from the experimenter-defined categories.

Because each experimenter-defined category in the \cite{jern2013probabilistic}
experiment possessed a distinct hue shared by all members of the category, it is
sensible that participants might generate a category with a hue distinct from
the experimenter-provided categories. If category generation were influenced by
category contrast in this way, the hues of generated categories should be
systematically different from those of the experimenter-defined categories.
Unfortunately, stimulus hue was encoded and presented in the
Hue-Saturation-Value (HSV) color space, which is device-dependent and not
perceptually normed such that perceived color similarity corresponds to
proximity in the color space (as opposed to a color space such as CIELAB that is
device-independent and equidistant sets of points correspond to pairs of colors
that have the same perceptual similarity; \citealp{wyszecki1967}). Further, they
did not calibrate their monitor, and so we cannot know the precise colors
presented to participants. As \cite{jern2013probabilistic} were interested in
relations between the saturation and length of examples in generated into novel
categories, these issues do not undercut their analyses and results. However,
these issues pose a significant challenge to evaluating contrast between the
experimenter-defined and participant-generated categories along the hue
dimension. It is plausible that two uncalibrated monitors could display the same
HSV color and the colors be perceived in different color categories (especially
for color boundaries that vary over lightness, such as the yellow-brown
boundary).

Although we cannot know the precise colors that were displayed or perceived, we
can still analyze their results from a coarse perspective to see whether there
is preliminary support for contrast. To do so, we binned all possible hues into
one of eight uniformly-spaced color groups: \{{\em Red}: $0-0.063$, $0.938-1$,
{\em Yellow}: $0.063-0.188$, {\em Yellow-Green}: $0.188-0.313$, {\em
	Green-Teal}: $0.313-0.438$, {\em Teal}: $0.438-0.563$, {\em Teal-Blue}:
$0.563-0.688$, {\em Purple}: $0.688-0.813$, {\em Pink}: $0.813-0.938$\}. In the
\cite{jern2013probabilistic} experiment, the hue of each experimenter-defined
category was selected from one of six possible values, each of which falls into
one of the color groups above (two color groups were not used as a possible hue
for the experimenter-defined categories). By categorizing the
participant-generated crystals likewise, we can obtain a broad measure of
category contrast by determining the proportion of participant-generated
crystals that fall into the same groups as the experimenter-defined categories:
If contrast influences the hues of the generated categories, we should observe
minimal overlap between in the color groupings.



\begin{figure}
	\begin{center} \inputpgf{figs/}{jk13-huecontrast.pgf}
		\caption{Analysis of data from \cite{jern2013probabilistic}, Experiment
			3. Plotted is the number of generated items that share a color group with one of
			the experimenter-defined classes. The ``Expected'' data follows a Binomial
			distribution with $p = 2/8 = 1/4$, given there were two experimenter-defined
			classes, and eight color groups.}
		\label{fig:jk13-huecontrast}
	\end{center}
\end{figure}

These data, shown in Figure \ref{fig:jk13-huecontrast}, reveal a clear pattern:
The majority of participants in each condition ($n = 22$) generated categories
possessing entirely distinct hues; with 0/6 exemplars sharing a hue with the
experimenter-defined categories. These results can be compared to the
predictions of a Binomial model, which proposes that participants generate hues
at random. That is, if hue selection is not systematic, the probability that any
given example will lie in the same color group as an experimenter-defined
category is given by a Binomial distribution with $p = 2/8=1/4$, as there were
two experimenter-defined categories and eight possible color groups. Chi-square
goodness-of-fit tests reveal that the observed distribution in each condition is
highly inconsistent with the hues being chosen at random, all
$\chi^2(6,N=22)>200$, $p < .001$. Participants tended to generate items that were
perceptually distinct from the categories they had learned, and were less likely
to generate hues possessed by members of the experimenter-defined categories.

It is possible, however, that this data can be explained by a process that does
not involve contrast. Specifically, since participants were trained on exemplars
that share the same hue within a category, it is plausible that they generate
exemplars that all share a common hue. If this were the case, given a $6/8 = .75$
probability that a hue distinct from the experimenter-defined categories is
selected, there is a corresponding $.75$ probability (or about 17 out of 22
participants) that the generated category will have no exemplars that share a
hue with any  of the experimenter-defined categories. In addition, that there is
a $2/8 = .25$ probability (about 6 out of 22 participants) that the generated
category will have all six exemplars that share a hue with any of the
experimenter-defined categories. This prediction of regularity across categories
is much closer to the data than the prediction from a Binomial model and does
not require any mechanism of category contrast.

However, upon analyzing the consistency of hues in the generated categories, it
is clear that participants do not generally produce categories with entirely
similar hues across all exemplars. Specifically, only 55\%  of all generated
categories comprised exemplars with a common hue. Consequently, the assumption
of regularity across categories may be inappropriate for this data set, leaving
the contrast-based explanation as the most plausible for our purposes.

Re-analyzing the results from \cite{jern2013probabilistic} provides some
corroborating support for contrast playing a role in category generation. Taken
alongside the analyses reported by \cite{jern2013probabilistic}, our analysis
suggests that generated categories tend to be distinct from {\em and}
distributionally similar to what is already known. However, it is worth noting
that our analysis is still limited: The color groups defined above are
imprecise, and it is not clear that our color grouping is consistent with the
psychological color boundaries perceived by participants. While we did obtain
similar results using a variety of alternative groupings, the hue dimension used
in the \cite{jern2013probabilistic} study does not lend itself straightforwardly
to the computation of similarities, and thus we cannot be certain of whether our
coding accurately approximates the psychological space of the stimuli. This
precludes traditional applications of categorization models to their data as it
is usually necessary to encode objects in psychological space in order to
accurately determine the similarity between objects. By consequence, although
these results likely indicate that contrast exerts {\em some} influence, they do
not precisely describe the nature of that influence. 

\bibliographystyle{apacite} \bibliography{citations.bib}

\end{document}

