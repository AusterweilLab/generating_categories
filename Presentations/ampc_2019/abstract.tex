% 
% Annual Cognitive Science Conference
% Sample LaTeX Paper -- Proceedings Format
% 

% Original : Ashwin Ram (ashwin@cc.gatech.edu)       04/01/1994
% Modified : Johanna Moore (jmoore@cs.pitt.edu)      03/17/1995
% Modified : David Noelle (noelle@ucsd.edu)          03/15/1996
% Modified : Pat Langley (langley@cs.stanford.edu)   01/26/1997
% Latex2e corrections by Ramin Charles Nakisa        01/28/1997 
% Modified : Tina Eliassi-Rad (eliassi@cs.wisc.edu)  01/31/1998
% Modified : Trisha Yannuzzi (trisha@ircs.upenn.edu) 12/28/1999 (in process)
% Modified : Mary Ellen Foster (M.E.Foster@ed.ac.uk) 12/11/2000
% Modified : Ken Forbus                              01/23/2004
% Modified : Eli M. Silk (esilk@pitt.edu)            05/24/2005
% Modified : Niels Taatgen (taatgen@cmu.edu)         10/24/2006
% Modified : David Noelle (dnoelle@ucmerced.edu)     11/19/2014
% Modified : Roger Levy (rplevy@mit.edu)     12/31/2018



%% Change "letterpaper" in the following line to "a4paper" if you must.

\documentclass[10pt,letterpaper]{article}

\usepackage{cogsci}

%\cogscifinalcopy % Uncomment this line for the final submission 

\usepackage{graphicx} %for figures
\usepackage{pslatex}
\usepackage{apacite}
\usepackage{float} % Roger Levy added this and changed figure/table
                   % placement to [H] for conformity to Word template,
                   % though floating tables and figures to top is
                   % still generally recommended!

%\usepackage[none]{hyphenat} % Sometimes it can be useful to turn off
%hyphenation for purposes such as spell checking of the resulting
%PDF.  Uncomment this block to turn off hyphenation.


%\setlength\titlebox{4.5cm}
% You can expand the titlebox if you need extra space
% to show all the authors. Please do not make the titlebox
% smaller than 4.5cm (the original size).
%%If you do, we reserve the right to require you to change it back in
%%the camera-ready version, which could interfere with the timely
%%appearance of your paper in the Proceedings.

%\title{[working title] Calling it what it is: Novel categories are distinct from Not-categories}

\title{}
 
\author{{\large \bf Shi Xian Liew (liew2@wisc.edu)} \\
  Department of Psychology, 1202 W. Johnson Street \\
  Madison, WI 53706 USA
  \AND {\large \bf Joseph L. Austerweil (austerweil@wisc.edu)} \\
  Department of Psychology, 1202 W. Johnson Street \\
  Madison, WI 53706 USA}


\begin{document}

\maketitle


\begin{abstract}

The topic of category generation is a growing
However, there is the general assumption that generating a new category B is the same as generating a new category
Not-A. generating a new category is equivalent to generating a category that is
not 

Understanding how new categories are generated and the factors influencing their structure is a growing area of interest
in cognitive science. One issue that has been overlooked, however, is the assumption that generating a distinct novel
category is equivalent to generating a category that is \emph{not} a known category. In this study, we provide a series
of preliminary work challenging this assumption by distinguishing two types of generated categories. Extending category
generation tasks from past research, we find that generated categories identified as \emph{not} what was known
consistently tend to be larger compared to generated categories that were identified with a specific label. We also
replicate a known effect demonstrating that generated categories share similar distributional properties as known
categories. Our findings provide the first steps in building a formal computational account of category generation.

\textbf{Keywords:} 
categorization; category generation; contrast; category learning; 
\end{abstract}

% \section{Acknowledgments}

% In the \textbf{initial submission}, please \textbf{do not include
%   acknowledgements}, to preserve anonymity.  In the \textbf{final submission},
% place acknowledgments (including funding information) in a section \textbf{at
% the end of the paper}.

\bibliographystyle{apacite}

\setlength{\bibleftmargin}{.125in}
\setlength{\bibindent}{-\bibleftmargin}

%\bibliography{citations.bib}


\end{document}
